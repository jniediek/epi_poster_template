%-*- mode: LaTeX; TeX-engine: xetex -*-
\documentclass{beamer}
\usetheme{epiukbbonn}

\usepackage[orientation=portrait, size=a3, scale=1]{beamerposter}

\usepackage{polyglossia}
\setmainlanguage{english}

\usepackage{fontspec}
\setsansfont{Arial}

\renewcommand{\emph}{\textbf}

%\beamertemplategridbackground[1cm]
%\setbeamertemplate{itemize item}{}

\setbeamertemplate{bibliography item}{}
\usepackage{siunitx}
\usepackage[
backend=biber,
style=authoryear,
maxnames=2,
maxbibnames=4,
url=false,
doi=false,
isbn=false,
firstinits=true
]{biblatex}


 \renewbibmacro*{date+extrayear}{%
     \iffieldundef{year}
       {}
       {\printtext[parens]{\printfield{labelyear}\printfield{extrayear}}}}%

% \DeclareNameAlias{byeditor}{sortname}
% % have last names first in all cases
% \DeclareNameFormat{sortname}{%
%    \ifuseprefix
%      {\usebibmacro{name:last-first}{#1}{#4}{#5}{#8}}
%      {\usebibmacro{name:last-first}{#1}{#4}{#6}{#8}}%
%    \usebibmacro{name:andothers}}
 \renewbibmacro*{in:}{}
% % Do not print numbers and eids
 \renewbibmacro*{volume+number+eid}{%
   \printfield{volume}}
 % Do not print notes
 \renewbibmacro*{note+pages}{%
   \setunit{\addcolon}%
   \printfield{pages}%
   \newunit}
% % Do not decorate the title
 \DeclareFieldFormat[article,book,inbook,incollection]{title}{\textit{#1}}
 \DeclareFieldFormat{journaltitle}{#1}
 \DeclareFieldFormat[article]{pages}{#1}
 \DeclareListFormat{language}{}
 \DeclareFieldFormat{booktitle}{#1}
\addbibresource{poster.bib}

\usecaptiontemplate{
\small
\structure{\insertcaptionname~\insertcaptionnumber:}
\insertcaption
} 


\linespread{1.02}
\newenvironment{wideitemize}{\itemize\addtolength{\itemsep}{.2em}\addtolength{\labelsep}{.1ex}}{\enditemize}

\title{Sample Title}
% this is sort of a hack, we use \date to provide the poster number
\date{P02}
\author{A. Author\textsuperscript{1}, V. A. Coenen\textsuperscript{2}, C. E. Elger\textsuperscript{1}, M. Soehle\textsuperscript{3} \& F. Mormann\textsuperscript{1}}
\institute{\textsuperscript{1}Dept. of Epileptology, \textsuperscript{2}Stereotaxy and MR based OR Techniques, Dept. of Neurosurgery, \textsuperscript{3}Dept. of Anaesthesiology and Intensive Care Medicine, University of Bonn, Germany}

\begin{document}
\begin{frame}[t]
\begin{columns}[T]
\begin{column}{.49\linewidth} % begin first column
\begin{block}{Overview}
\begin{columns}[T]
\begin{column}{.34\linewidth}
\emph{Background}\vspace{\itemsep}
\begin{wideitemize}
\item Extracellular micro-electrode recordings from the human brain have been feasible for some years
\item \emph{Spike-sorting} is necessary to identify contributions of individual neurons from raw recordings
\item Multi-hour single-unit recordings would allow to study important questions regarding both epilepsy (e.g., ictogenesis \parencite{gast_burst_2016}) and cognition (e.g., memory consolidation)
\item \emph{However}, no tools for the spike-sorting of multi-hour single-unit recordings have been available
\end{wideitemize}
\end{column}
\begin{column}{.57\linewidth}
\emph{Approach}
\begin{wideitemize}
\item We developed \emph{Combinato}, a data analysis framework for multi-hour single-unit recordings (\cite{niediek_reliable_2016}). This framework handles
  \begin{wideitemize}
  \item Artifact rejection
  \item Automatic spike-sorting
  \item Guided manual improvement of spike-sorting
  \end{wideitemize}
  \item Combinato is based on  spike-sorting in blocks followed by combination of clusters from blocks. Each block is sorted by superparamagnetic clustering \parencite{quian_quiroga_unsupervised_2004}.
  \end{wideitemize}
  
\emph{Evaluation}
\begin{wideitemize}
  \item When tested on simulations, Combinato outperformed manual operators of the wide-spread package WaveClus
  \item On data recorded from the medial temporal lobe of patients to find selective single-unit responses, Combinato again outperformed WaveClus
\item Combinato was able to track selectively responding single-unit throughout entire nights
\end{wideitemize}

\end{column}
\end{columns}
\end{block}


  \begin{block}{Results: Simulated model data}

\begin{columns}[T]
\begin{column}{.95\linewidth}
\emph{Evaluation on 10-minute simulations}
        \begin{wideitemize}
    \item We used published simulated model data \parencite{pedreira_how_2012}, consisting of 95~simulations of 10~minutes duration each.
    \item Spike-sorting performance was measured by the number of ``hits'', where a hit is a unit consisting of at least 50\% of the spikes of a true unit, and with at most 50\% contamination.
    \item We compared Combinato's results to published results \parencite{pedreira_how_2012} obtained by manual operators of WaveClus \parencite{quian_quiroga_unsupervised_2004}.
    \end{wideitemize}

  
\begin{columns}[T]
  \begin{column}{.47\linewidth}

    \begin{figure}
\label{fig:results_sim}
\begin{center}
%\includegraphics[width=\linewidth]{fig3.png}
\end{center}
\caption{\emph{Results for one simulated channel.} \textbf{A}~Automatically selected clusters at different ``temperatures''. \textbf{B}~Performance at default parameters. \textbf{C}~Clusters generated in one simulation. \textbf{D}~Manually splitting unit C1 from C generated another hit.}
\end{figure}
\end{column}
\begin{column}{.48\linewidth}


  \begin{figure}
\label{fig:results_sim}
\begin{center}
%\includegraphics[width=\linewidth]{params.png}
\end{center}
\caption{\emph{Performance at different parameter settings.} Displayed are results for 8~of the 48~parameter settings tested. Asterisks indicate that the percentage of hits is significantly higher than the optimum obtained by \cite{pedreira_how_2012}. The fully automated Combinato outperformed manual operators of WaveClus at 24 of the 48 settings, indicating robustness of our algorithms. \textsuperscript{***}, P < .001; \textsuperscript{**}, P < .01, \textsuperscript{*}, P < .05; n.s., better but not significantly.}
\end{figure}

\end{column}
\end{columns}

\vspace{2em}
\emph{Evaluation on 10-hour simulations}
\begin{wideitemize}
\item We concatenated simulations of 10~minutes duration to produce 19 simulations of 10~hours duration.
\item To make the task harder, we added drift and Gaussian noise to the concatenated data.
\end{wideitemize}
\begin{columns}[T]
\begin{column}{.47\linewidth}
\begin{figure}
\begin{center}
%\includegraphics[width=\linewidth]{fig5.png}
\end{center}
\caption{\emph{Simulated neuron with drift and added noise.} The unmodified unit is shown in orange in the bottom panel for comparison.}
\end{figure}
\end{column}

\begin{column}{.47\linewidth}
\begin{figure}
\begin{center}
%\includegraphics[width=.7\linewidth]{fig6.png}
\end{center}
\caption{\emph{Performance at different parameter settings.} \textbf{A}~Numbers of hits for different numbers of units in the simulated data. Four parameter settings were tested. \textbf{B}~Mean and standard deviation for each paramter setting.}
\end{figure}
\vspace{-1.6em}
\end{column}
\end{columns}
\end{column}
\end{columns}
\end{block}


\end{column} %end second poster column


\begin{column}{.49\linewidth} % begin second poster column
\begin{block}{Methods \& Software}
 % \vspace{-.5em}
  \begin{columns}[T]
    \begin{column}{.48\linewidth}
      \emph{Automated data processing framework}
      \vspace{-1em}
  \begin{figure}
%    \includegraphics[width=\textwidth]{fig1.png}
    \caption{\textbf{Schematic of data processing.} \textbf{A}~Channels without unit activity are excluded. Action potentials are detected after bandpass filtering. \textbf{B}~Detected artifact events are excluded according to three criteria. \textbf{C}~Spike-sorting is performed in parallel in blocks of \num{20000} by ``superparamagnetic clustering'' (\cite{quian_quiroga_unsupervised_2004}). In each block, a second clustering step is  applied to all large clusters. \textbf{D}~Remaining spikes are assigned by Euclidean template matching. \textbf{E}~Artifact clusters are removed. \textbf{F}~Similar clusters are grouped both within each block and across blocks. }
  \end{figure}
\end{column}
\begin{column}{.42\linewidth}
  \emph{Graphical user interfaces}
  \vspace{-1em}
  \begin{figure}
%  \includegraphics[width=.9\textwidth]{css-overview-screenshot.png}\\[1em]
\caption{\emph{Combinato's data viewer.} Both the raw signal (at several temporal resolutions) as well as the clustering result for each channel are displayed.}
\end{figure}
\vspace{-1.9em}
\begin{figure}
%\includegraphics[width=.9\textwidth]{css-gui-screenshot.png}
\caption{\emph{Combinato's tool for result inspection.} Several graphical quality metrics are displayed for each cluster. Clusters can be manually discarded or manipulated.}
\end{figure}
\vspace{-2em}
\end{column}
\end{columns}

\end{block}

\begin{block}{Results: Recordings from epilepsy patients}
\begin{columns}[T]
  \begin{column}{.95\linewidth}
    \vspace{-.5em}
\emph{Micro-electrode recordings}
\begin{wideitemize}
\item Micro-electrode recordings from 10~neurosurgical patients undergoing epilepsy monitoring implanted with intracerebral micro-electrodes (as in \cite{mormann_neurons_2015}).  Electrodes were located bilaterally in the hippocampus, amygdala, entorhinal cortex, and parahippocampal cortex.

\end{wideitemize}

%\vspace{-1.6em}
  \begin{columns}[T]
    \begin{column}{.48\linewidth}
\begin{figure}[T]
  \begin{center}
%    \includegraphics[width=\linewidth]{fig7.png}
\end{center}
\caption{\emph{Result of automatic sorting on one channel with noise.} \textbf{A}~Bandpass filtered data. \textbf{B}~Pre-sorting artifact removal. \textbf{C}~Automatically selected clusters at different temperatures. \textbf{D}~Clusters and artifacts as identified by Combinato. \textbf{E}~Clusters as identified by WaveClus. \textbf{F}~Raster plots from visual stimulus presentation. Cluster D2 responds to all four pictures. Three of the four responses remain undetected when using WaveClus.}
\end{figure}
%\vspace{-2em}

\end{column}

\begin{column}{.48\linewidth}
\begin{figure}[T]
%  \includegraphics[width=.95\textwidth]{fig10.png}
  
\caption{\emph{Examples of neurons tracked over an entire night.}
\textbf{A}~Stable waveform and response pattern.
\textbf{B}~Amplitude variations; no response to the written name.
\textbf{C}~Amplitude shift results in the detection of two different units.
\textbf{D}~Two resp.~cclusters generated; no response to the written name.
\textbf{E}~Stable waveform; weak response in the morning.
\textbf{F}~Solid response in the evening and morning, but with separate units. 
\textbf{G}~The blue and red clusters both contribute to the response; both tracked with a stable waveform.
\textbf{H}~The red, blue, and yellow clusters contribute to the response. All three clusters have a stable waveform.
}

\end{figure}
%\vspace{-2em}
\end{column}
\end{columns}

\end{column}
\end{columns}
\end{block}
\begin{block}{References}
\begin{columns}[T]
  \begin{column}{.97\linewidth}
%    \vspace{-.5em}
%    \emph{References}
\setbeamercolor{normal text}{fg=black}
\setbeamercolor{structure}{fg=black}
\appto\bibfont{\small}
\printbibliography
%\emph{Funding}\\
%\small
%This work was supported by Volkswagen Foundation (Lichtenberg 86 507); German Research Council (DFG MO 930/4-1, \emph{SFB 1089}); Seventh Framework Program (602102 EPITARGET).
\end{column}
\end{columns}
\vspace{.9em}
\end{block}

\end{column} % third poster column
\end{columns}

\begin{columns}[c]% this is the footer
  \begin{column}{.3\linewidth} 
    \includegraphics[width=\linewidth]{../figures/logo-vwstiftung.png}
  \end{column}
  \begin{column}{.3\linewidth}
    \begin{centering}
      \textit{
     \hfill Lichtenberg Evaluation, Oct 12, 2017 \hfill}
    \end{centering}
  \end{column}
  \begin{column}{.3\linewidth}
    \raggedleft{\textit{
    Contact:\\
    \texttt{jonied@posteo.de}}}
  \end{column}
\end{columns}
\end{frame}
\end{document}
